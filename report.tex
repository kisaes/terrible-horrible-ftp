% This is LLNCS.DEM the demonstration file of the LaTeX macro package from
% Springer-Verlag for Lecture Notes in Computer Science, version 2.4 for
% LaTeX2e as of 16. April 2010
%
\documentclass{llncs}
%
\begin{document}
%
%
\pagestyle{headings}  % switches on printing of running heads
%
\title{FTP Distribuido}
%
\author{}
%
\institute{}

\maketitle              % typeset the title of the contribution
% \renewcommand{\abstractname}{Resumen}

\section*{Arquitectura o el problema de como diseñar el sistema}

Cualquier arquitectura que garantice la comunicación directa entre nodos es válida, la elección va a depender
en mayor medida de la facilidad del proceso de descubrimiento de nodos que de la operación misma del sistema
distribuido. Se necesita poder consultar que nodos están en línea y consultar si un nodo específico está en línea.

\section*{Procesos o el problema de cuantos programas o servicios posee el sistema}

Un servidor ftp consiste en solamente un proceso en espera de clientes y multiples sesiones (una sesión por cliente),
el funcionamiento del servidor y las sesiones es multiplexado en solamente un hilo, de modo que la creación de nuevas
sesiones y el comienzo de la ejecución de los comandos ftp son realizados de forma síncrona y en orden de llegada,
casi todos los comandos ftp (excepto los de transferencia de datos) terminan inmediatamente, el resto provoca la creación
de un hilo que completa la transferencia de datos y la ejecución del comando ftp sin interferir con el funcionamiento
normal del servidor y el resto de sesiones. De esta forma un servidor ftp soporta multiples endpoints, multiples clientes,
multiples sesiones por cliente y multiples operaciones concurrentes por cliente (incluso si esto último no está definido
en la especificación del protocolo).

\section*{Comunicación o el problema de como enviar información mediante la red}

La comunicación cliente - servidor está definida por el protocolo. El servidor consiste en un solo proceso (no existe IPC).
Cada servidor asume que la comunicación directa con el resto de servidores es posible y se realiza intercambiando mensajes
en cualquier formato predefinido.

\section*{Coordinación o el problema de poner todos los servicios de acuerdo}

Cada servidor ftp es responsable de sus archivos locales únicamente, la toma de decisiones no es distribuida, cada nodo
decide si aceptar peticiones de otros servidores de acuerdo a que acciones son válidas en sus archivos locales. El acceso
concurrente a un archivo (local o no) solo está permitido para operaciones de lectura.

\section*{Nombrado y localización o el problema de dónde se encuentra un recurso y como llegar al mismo}

La estructura de archivos y directorios está bien definida y la ubicación de los recursos en la red viene implícita en su ubicación
en el árbol conjunto de directorios.

\section*{Consistencia y replicación o el problema de solucionar los problemas que surgen a partir de tener varias copias de un mismo dato en el sistema.}

Los archivos se almacenan exclusivamente en su raíz local, lo que significa que durante una desconexión el acceso se limita
a los archivos locales de los nodos alcanzables. Lamentablemente la única forma de impedir la pérdida de acceso es replicar
todos los archivos en todos los nodos porque no podemos predecir que archivo va a necesitar el usuario ni como ocurrirá la
desconexión.

\section*{Tolerancia a fallas o el problema de, para que pasar tanto trabajo distribuyendo datos y servicios si al fallar una componente del sistema todo se viene abajo.}

En caso de desconexión cada nodo siempre va a poder acceder como mínimo a sus archivos locales, por tanto las operaciones
con los archivos locales solamente fallan con la desconexión directa de cliente. Además, el protocolo ftp especifica
que ante la desconexión del socket de control las operaciones deben abortarse, lo que significa en nuestro caso que
incluso si un archivo no local es accesible, la transferencia debe abortarse si el servidor local se desconecta.

\section*{Seguridad o el problema de que tan vulnerable es su diseño}

El protocolo ftp es inseguro por diseño, la comunicación es en texto plano y la autenticación y autorización son definidos
por la plataforma (generalmente contraseña, de igual forma transmitida en texto plano). Posibles mejoras incluyen
utilizar TLS (RFC 2228 FTP Security Extensions) y utilizar algún mecanismo propio de autenticación.

\end{document}
